
%%%%%%%%%%%%%%%%%%%%%%%%%%%%%%%%%%%%%%%%%%%%%%%%%%%%%%%%%%%%%%%
%%%%%%%%%%%%%%%%%%%%%%%%%%%%%%%%%%%%%%%%%%%%%%% Getting Started

\section{Getting Started}
\begin{frame}
    \vspace{25mm}
    \begin{center}
        \Huge{Part 1:\\Getting Started}
    \end{center}
\end{frame}

\subsection{Two Modes of Execution}
{
\usebackgroundtemplate{\includeSVGfs{execution}}
\begin{frame}
    \frametitle{Two Modes of Execution}
    \vspace{5mm}
    
\end{frame}
}

\subsection{Imports}
\defverbatim[colored]\contentImports{
\begin{minted}{python}
import os
from sys import argv
from sys import exit as bye

print(os.name)
bye()
\end{minted}
}
\begin{frame}
    \frametitle{Imports}
    \vspace{3mm}
    \contentImports
\end{frame}

\subsection{First Steps}
\defverbatim[colored]\contentFirstSteps{
\begin{minted}{python}
#!/usr/bin/env python3
import sys

print("Hello, world!")
sys.exit() # this is really not necessary
\end{minted}
}
\begin{frame}
    \frametitle{First Steps}
    \vspace{3mm}
    \contentFirstSteps
\end{frame}

\subsection{Command-Line Arguments}
\defverbatim[colored]\contentCLA{
\begin{minted}{python}
from sys import argv, exit

if len(argv) != 3:
    print('Syntax: %s INPUT_FILENAME OUTPUT_FILENAME' % argv[0])
    print('        %s log.txt analysis.csv' % argv[0])
    exit(1)
input_filename  = argv[1]
output_filename = argv[2]

print('%s -> %s' % (input_filename, output_filename))
\end{minted}
}
\defverbatim[colored]\contentCLAoutput{
\begin{minted}[fontsize=\tiny]{shell-session}
aslak@thera:~/vcs/git/dm-course/src/python$ python3 process.py 
Syntax: process.py INPUT_FILENAME OUTPUT_FILENAME
        process.py log.txt analysis.csv
aslak@thera:~/vcs/git/dm-course/src/python$ python3 process.py log.txt analysis.csv
log.txt -> analysis.csv
aslak@thera:~/vcs/git/dm-course/src/python$
\end{minted}
}
\begin{frame}
    \frametitle{Command-Line Arguments}
    \vspace{3mm}
    \contentCLA
    
    \pause
    \vspace{7mm}
    \contentCLAoutput
\end{frame}

%%%%%%%%%%%%%%%%%%%%%%%%%%%%%%%%%%%%%%%%%%%%%%%%%%%%%%%%%%%%%%%
%%%%%%%%%%%%%%%%%%%%%%%%%%%%%%% Basic Datatypes and -Structures

\section{Basic Datatypes and -Structures}
\begin{frame}
    \vspace{25mm}
    \begin{center}
        \Huge{Part 2:\\Basic Datatypes and -Structures}
    \end{center}
\end{frame}

\subsection{Types}
{
\usebackgroundtemplate{\includeSVGfs{types}}
\begin{frame}
    \frametitle{Types}
\end{frame}
}

\subsection{Boolean Operators}
\defverbatim[colored]\contentBooleans{
\begin{minted}{python}
if page < pagecount and good_book:
    print('Enjoy!')

while not there:
    print('Are we there yet?')

if finished or not started:
    print('Not much is happening :-(')
\end{minted}
}
\begin{frame}
    \frametitle{Boolean Operators}
    \vspace{5mm}
    In python boolean operators are spelled out.
    
    \vspace{5mm}
    \contentBooleans
\end{frame}

\subsection{String Operations}
\defverbatim[colored]\contentStrings{
\begin{minted}[fontsize=\tiny]{pycon}
>>> s1 = "Alice was beginning to get very tired of sitting by her sister on the bank"
>>> s1
'Alice was beginning to get very tired of sitting by her sister on the bank'
>>> s2 = 'and of having nothing to do'
>>> s2
'and of having nothing to do'
>>> s = s1+", "+s2
>>> s
'Alice was beginning to get very tired of sitting by her sister on the bank, and of having nothing to do'
>>> s = s.replace(',', '')
>>> s
'Alice was beginning to get very tired of sitting by her sister on the bank and of having nothing to do'
>>> words = s.split(' ')
>>> words
['Alice', 'was', 'beginning', 'to', 'get', 'very', 'tired', 'of', 'sitting', 'by', 'her', 'sister', 'on', 'the',
'bank', 'and', 'of', 'having', 'nothing', 'to', 'do']
>>> '%d: %s' % (3, 'March')
'3: March'
>>> '_'.join(words)
'Alice_was_beginning_to_get_very_tired_of_sitting_by_her_sister_on_the_bank_and_of_having_nothing_to_do'
\end{minted}
}
\begin{frame}
    \frametitle{String Operations}
    \vspace{5mm}
    \contentStrings
\end{frame}

\defverbatim[colored]\contentFunction{
\begin{minted}{python}
def add (a, b, c=0):
    return a+b+c

print(add(1,2,3))
print(add(1,2))

a = add
print(a(1,2))
\end{minted}
}
\subsection{Functions}
\begin{frame}[fragile]
    \frametitle{Functions}
    \vspace{-2mm}
    \contentFunction
    \pause
    \vspace{5mm}
    \begin{verbatim}
6
3
3
    \end{verbatim}
\end{frame}

\defverbatim[colored]\contentTypeIntrospectionI{
\begin{minted}{pycon}
>>> t = type(True)
>>> t
<class 'bool'>
>>> type(t)
<class 'type'>
>>> type(bool)
<class 'type'>
>>> t == bool
True
\end{minted}
}
\defverbatim[colored]\contentTypeIntrospectionII{
\begin{minted}{pycon}
>>> def fun(var): return var
... 
>>> type(fun)
<class 'function'>
>>> f = fun
>>> type(f)
<class 'function'>
>>> f(1)
1
>>> g = lambda a: a
>>> type(g)
<class 'function'>
>>> g(1)
1
\end{minted}
}
\subsection{Type Introspection}
\begin{frame}
    \frametitle{Type Introspection}
    \vspace{0mm}
    \contentTypeIntrospectionI
\end{frame}
\begin{frame}
    \frametitle{Type Introspection}
    \vspace{0mm}
    \contentTypeIntrospectionII
\end{frame}

\subsection{Object-Orientation}
\begin{frame}
    \frametitle{Object-Orientation}
    \vspace{2mm}
    
    \inputminted[fontsize=\small]{python}{../src/python/objectexample.py}
\end{frame}

%%%%%%%%%%%%%%%%%%%%%%%%%%%%%%%%%%%%%%%%%%%%%%%%%%%%%%%%%%%%%%%
%%%%%%%%%%%%%%%%%%%%%%%%%%%%%%%%%%%%%%%%%%%%%%%%%% Flow Control

\section{Flow Control}
\begin{frame}
    \vspace{25mm}
    \begin{center}
        \Huge{Part 3:\\Flow Control}
    \end{center}
\end{frame}

\subsection{Branching}
\defverbatim[colored]\contentBranching{
\begin{minted}{python}
if len(lines)>0 and len(line[0])>0 and line[0][0]=='#':
    print('First line is a comment')

parts = line.split(' ')
command = parts[0]
if   command=='load':
    load_file()
elif command=='save':
    save_file()
elif command=='quit':
    quit()
else:
    print('Unknown command "'+command+'"')
\end{minted}
}
\begin{frame}
    \frametitle{Branching}
    \vspace{3mm}
    \contentBranching
\end{frame}

\subsection{Missing For-Loop}
\defverbatim[colored]\contentForI{
\begin{minted}{python}
for line in lines:
    print(line)
\end{minted}
}
\defverbatim[colored]\contentForII{
\begin{minted}{python}
for i in range(len(lines)):
    line = lines[i]
    print(str(i)+': '+line)
\end{minted}
}
\begin{frame}
    \frametitle{Missing For-Loop}
    \vspace{3mm}
    Python does not have a \textsl{for} loop.
    
    \pause
    \vspace{3mm}
    Python has a \textsl{foreach} loop.
    \pause
    
    \vspace{6mm}
    Iterating over a list:
    \vspace{3mm}
    \contentForI
    \pause
    
    \vspace{6mm}
    Iterating over a list with access to the index:
    \vspace{3mm}
    \contentForII
\end{frame}

\subsection{Generating Ranges of Integers}
\defverbatim[colored]\contentRange{
\begin{minted}{pycon}
>>> range(5)
range(0, 5)
>>> list(range(5))
[0, 1, 2, 3, 4]
>>> list(range(1,5))
[1, 2, 3, 4]
>>> list(range(1,5,2))
[1, 3]
>>> for i in range(1,5,2):
...     print(i)
... 
1
3
\end{minted}
}
\begin{frame}
    \frametitle{Generating Ranges of Integers}
    \vspace{3mm}
    The \texttt{range} function returns a generator for a sequence of integers.
    \pause
    \vspace{3mm}
    \contentRange
\end{frame}

%%%%%%%%%%%%%%%%%%%%%%%%%%%%%%%%%%%%%%%%%%%%%%%%%%%%%%%%%%%%%%%
%%%%%%%%%%%%%%%%%%%%%%%%%%%%%%%%%%%%%%%%%%%%%%%%%%%%%%%%% Lists

\section{Lists}
\begin{frame}
    \vspace{25mm}
    \begin{center}
        \Huge{Part 4:\\Lists}
    \end{center}
\end{frame}

\subsection{Basic Operations}
\defverbatim[colored]\contentList{
\begin{minted}[fontsize=\tiny]{pycon}
>>> l = [1,2,3]
>>> l
[1, 2, 3]
>>> l.append(4)
>>> l
[1, 2, 3, 4]
>>> l.extend([7,6,5])
>>> l
[1, 2, 3, 4, 7, 6, 5]
>>> sorted(l)
[1, 2, 3, 4, 5, 6, 7]
>>> l
[1, 2, 3, 4, 7, 6, 5]
>>> l.sort()
>>> l
[1, 2, 3, 4, 5, 6, 7]
>>> len(l)
7
>>> l[2], l[-1]
(3, 7)
>>> l[2:]
[3, 4, 5, 6, 7]
>>> l[2:4]
[3, 4]
>>> l[:4]
[1, 2, 3, 4]
>>> 4 in l, 42 in l
(True, False)
\end{minted}
}
\begin{frame}
    \frametitle{List Operations}
    \vspace{3mm}
    \contentList
\end{frame}

\subsection{Higher-Order Functions}
\defverbatim[colored]\contentLambda{
\begin{minted}[fontsize=\normalsize]{pycon}
>>> l = [-17,2,5,-4,4,7,-3,-1,9,1]
>>> incr = lambda v: v+1
>>> map(incr, l)
<map object at 0x7f33c8440b20>
>>> list(map(incr, l))
[-16, 3, 6, -3, 5, 8, -2, 0, 10, 2]
>>> pos = lambda v: v>=0
>>> filter(pos, l)
<filter object at 0x7f33c8440b20>
>>> list(filter(pos, l))
[2, 5, 4, 7, 9, 1]
>>> list(map(incr, filter(pos, l)))
[3, 6, 5, 8, 10, 2]
\end{minted}
}
\begin{frame}
    \frametitle{Higher-Order Functions over Lists}
    \vspace{0mm}
    \contentLambda
\end{frame}


%%%%%%%%%%%%%%%%%%%%%%%%%%%%%%%%%%%%%%%%%%%%%%%%%%%%%%%%%%%%%%%
%%%%%%%%%%%%%%%%%%%%%%%%%%%%%%%%%%%%%%%%%%%%%%%%%% Dictionaries

\section{Dictionaries}
\begin{frame}
    \vspace{25mm}
    \begin{center}
        \Huge{Part 5:\\Dictionaries}
    \end{center}
\end{frame}

\subsection{Basic Operations}
\defverbatim[colored]\contentDict{
\begin{minted}[fontsize=\tiny]{pycon}
>>> {}
{}
>>> d = {'jan': 1, 'feb': 2, 'mar': 3}
>>> d
{'jan': 1, 'feb': 2, 'mar': 3}
>>> d['jan']
1
>>> d['apr'] = 4
>>> d
{'jan': 1, 'feb': 2, 'mar': 3, 'apr': 4}
>>> d['list'] = [1,2,3]
>>> d
{'jan': 1, 'feb': 2, 'mar': 3, 'apr': 4, 'list': [1, 2, 3]}
>>> 'jan' in d
True
>>> 'may' in d
False
>>> d.keys()
dict_keys(['jan', 'feb', 'mar', 'apr', 'list'])
>>> list(d.keys())
['jan', 'feb', 'mar', 'apr', 'list']
>>> for key in d: print(key)
... 
jan
feb
mar
apr
list
>>> del(d['feb'])
>>> d
{'jan': 1, 'mar': 3, 'apr': 4, 'list': [1, 2, 3]}
\end{minted}
}
\begin{frame}
    \frametitle{Basic Operations}
    \vspace{0mm}
    \contentDict
\end{frame}

%%%%%%%%%%%%%%%%%%%%%%%%%%%%%%%%%%%%%%%%%%%%%%%%%%%%%%%%%%%%%%%
%%%%%%%%%%%%%%%%%%%%%%%%%%%%%%%%%%%%%%%%%%%%%%%%%%%%%%% Strings

\section{Strings}
\begin{frame}
    \vspace{25mm}
    \begin{center}
        \Huge{Part 6:\\Strings}
    \end{center}
\end{frame}

\subsection{Basic Operations}
\defverbatim[colored]\contentStrings{
\begin{minted}{pycon}
>>> initial = ' once upon a time  '
>>> initial
' once upon a time  '
>>> len(initial)
19
>>> stripped = initial.strip()
>>> stripped
'once upon a time'
>>> words = stripped.split(' ')
>>> words
['once', 'upon', 'a', 'time']
>>> words[1], stripped[1]
('upon', 'n')
>>> joined = '_'.join(words)
>>> joined
'once_upon_a_time'
\end{minted}
}
\begin{frame}
    \frametitle{\textbf{Strings:} Basic Operations}
    \vspace{0mm}
    \contentStrings
\end{frame}

\subsection{Regular Expressions}
\defverbatim[colored]\contentRegexp{
\begin{minted}[fontsize=\footnotesize]{python}
import re

urls = [
    'https://www.gutenberg.org/files/11/11-h/11-h.htm',
    'https://golang.org',
    'http://www.google.com:80/',
    'definitely not a URL',
]
pattern = re.compile('([^:]+)://([^:/]+)(:\d+|)(/.*|)')

for url in urls:
    mo = pattern.match(url)
    
    if mo:
        print('proto="%s" domain="%s" port="%s" path="%s"' %
              (mo.group(1), mo.group(2), mo.group(3), mo.group(4)))
\end{minted}
}
\begin{frame}[fragile]
    \frametitle{Regular Expressions}
    \vspace{0mm}
    \contentRegexp
    
    \vspace{0mm}
    \pause
    {\footnotesize
    \begin{verbatim}
proto="https" domain="www.gutenberg.org" port="" path="/files/11/11-h/11-h.htm"
proto="https" domain="golang.org" port="" path=""
proto="http" domain="www.google.com" port=":80" path="/"
    \end{verbatim}
    }
\end{frame}

\section{More Information}
\begin{frame}
    \vspace{25mm}
    \begin{center}
        {\Huge Hungry for more?}\\ \vspace{10mm}
        \url{https://github.com/aslakjohansen/enigma-python-intro}
    \end{center}
\end{frame}

%%%%%%%%%%%%%%%%%%%%%%%%%%%%%%%%%%%%%%%%%%%%%%%%%%%%%%%%%%%%%%%
%%%%%%%%%%%%%%%%%%%%%%%%%%%%%%%%%%%%%%%%%%%%%%%%%%%%% Questions

\section{Questions}
\begin{frame}
    \frametitle{\textbf{Questions and Comments?}}
    \vspace{-15mm}
    \begin{center}
    \includegraphics[scale=0.4]{./figs/Boy-asking-question.pdf}
    \end{center}
    \vspace{-25mm}
    \scalebox{0.2}{\url{https://openclipart.org/detail/238687/boy-thinking-of-question}}
\end{frame}

